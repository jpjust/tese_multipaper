\begin{thesisabastract}

The accelerated growth of urban centres presents complex challenges for planning safer cities. Population growth, intense vehicle traffic and structural and socioeconomic inequalities are some of the factors that increase the risks faced by the population. In this scenario, urban emergencies can arise at any time, with diverse impacts, making rescuing victims and mitigating such emergencies quite challenging. To deal with this reality, open georeferenced data can provide significant information about the dynamics of urban systems and their population. In this context, this research main objective is to develop a set of computational solutions, algorithms and processes to identify and assess risk areas in urban environments, using open data. Thus, the aim is to create a comprehensive methodology that can support the planning and sustainable development of any urban centre, leveraging the development of resilient cities. The results obtained include the development of an effective methodology for classifying urban risks as well as the implementation of a \textit{web} platform that makes the proposed solution openly available. Furthermore, innovative algorithms for positioning sensing units were developed using the risk modeling created. In this way, authorities and organizations will have important information about the most vulnerable areas and how to better treat them, bringing countless benefits to a scenario of growing challenges brought by climate changes.\\

\textbf{Keywords:} \ppgmkeywords

\end{thesisabastract}
