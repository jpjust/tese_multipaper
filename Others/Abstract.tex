\begin{thesisabastract}

The growth of big urban centers have brought new challenges in cities planning. The demographic growth, the high vehicles traffic, the constant new buildings, the rise of the number of commercial centers, among other factors, increase the risk in citizens daily living. Emergencies and crisis may happen and, because of the great concentration of people and vehicles, rescuing and mitigation become new challenges in the modern world. These challenges demand the use of smart cities technologies and applications as solutions for crisis management in big urban centers. In front of this problem, this doctorate research's goal was to produce a set of computational solutions, algorithms and procedures to perform risk to emergencies assessment in urban areas, thus, providing a generalist methodology, supporting the development of the concept of smart cities and contributing to seize the Sustainable Development Goals of Agenda 2030 of United Nations. The results of this research were the development of a risk assessment approach in cities, the development of a web tool that allows the use of the proposed approach and the application of sensing units positioning algorithms from the results of the risk assessment.\\

\textbf{Keywords:} \ppgmkeywords

\end{thesisabastract}
