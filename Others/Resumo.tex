\begin{thesisresumo}

O crescimento acelerado dos centros urbanos apresenta desafios complexos para o planejamento de cidades mais seguras. O aumento populacional, o intenso tráfego de veículos e desigualdades estruturais e socioeconômicas são alguns dos fatores que ampliam os riscos enfrentados pela população. Nesse cenário, emergências urbanas podem surgir a qualquer momento, com impactos diversos, tornando bastante desafiante o resgate de vítimas e a mitigação de tais emergências. Para lidar com essa realidade, dados georreferenciados abertos podem fornecer informações significativas sobre a dinâmica dos sistemas urbanos e da sua população. Diante desse contexto, esta pesquisa de doutorado tem como principal objetivo desenvolver um conjunto de soluções computacionais, algoritmos e processos para identificar e classificar áreas de risco em ambientes urbanos, utilizando dados abertos. Assim, pretende-se criar uma metodologia abrangente que possa apoiar o planejamento e o desenvolvimento sustentável de qualquer centro urbano, fomentando o desenvolvimento de cidades resilientes. Os resultados obtidos incluem o desenvolvimento de uma metodologia eficaz para a classificação de riscos urbanos, assim como a implementação de uma plataforma \textit{web} que disponibiliza abertamente a solução proposta. Além disso, algoritmos inovadores para o posicionamento de unidades de sensoriamento foram desenvolvidos utilizando a modelagem de risco criada. Dessa forma, autoridades e organizações adquirem importantes informações sobre áreas mais vulneráveis e como melhor tratá-las, trazendo inúmeros benefícios para um cenário de crescentes desafios marcado por mudanças climáticas.\\

\textbf{Palavras Chaves:} \ppgmpalavraschave

\end{thesisresumo}
