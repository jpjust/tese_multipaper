\begin{thesisresumo}

O crescimento dos grandes centros urbanos tem trazido novos desafios para o planejamento das cidades. O aumento demográfico, o alto tráfego de veículos, as constantes construções, o crescente número de centros comerciais, dentre outros fatores, trazem consigo o aumento do risco no dia a dia dos cidadãos. Emergências e crises podem ocorrer e, devido a essa grande concentração de pessoas e veículos, o resgate e a mitigação tornam-se novos desafios no mundo moderno. Esses desafios exigem o uso das tecnologias e aplicações de cidades inteligentes como soluções para o gerenciamento de crises em grandes centros urbanos. Diante deste problema, esta pesquisa de doutorado teve como objetivo produzir um conjunto de soluções computacionais, algoritmos e processos para classificação de zonas de risco a emergências em áreas urbanas, fornecendo assim uma metodologia generalista, dando suporte ao desenvolvimento do conceito de cidades inteligentes e contribuindo para alcançar as metas dos Objetivos de Desenvolvimento Sustentável da Agenda 2030 da Organização das Nações Unidas. A pesquisa obteve como resultados o desenvolvimento de uma abordagem de classificação de riscos em cidades, produzindo também uma ferramenta \textit{web} para uso aberto da abordagem proposta, bem como aplicações através de algoritmos de posicionamento de unidades de sensoriamento a partir do resultado de uma classificação.\\

\textbf{Palavras Chaves:} \ppgmpalavraschave

\end{thesisresumo}
