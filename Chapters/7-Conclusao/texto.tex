% --------------------------------------------------
% Capítulo 7: Considerações finais
% --------------------------------------------------
\chapter{Considerações Finais}\label{cap:conclusao}

Gerenciamento de emergências é um tópico bastante relevante e de grande interesse por gestores e autoridades de grandes centros urbanos. Conforme apresentado no Capítulo~\ref{cap:survey}, diversos pesquisadores trabalham neste assunto com o objetivo de desenvolver abordagens e soluções para detecção, alerta e mitigação de eventos de emergência em cidades inteligentes, fazendo uso de tecnologias da informação e comunicação em geral. Porém, a maioria das soluções propostas atacam um cenário específico, seja um evento de emergência ou uma cidade em particular.

Esta pesquisa de doutorado propôs uma abordagem generalista, que não considera nenhum evento de emergência específico nem uma região em particular. Usando como ponto de partida a localização de centros de resposta a emergências, a abordagem de classificação de risco proposta no Capítulo~\ref{cap:posicionamento} é capaz de segmentar a área de interesse a qual se deseja classificar em zonas de mitigação, definindo então os níveis de mitigação de cada zona de acordo com a presença e distância de centros de resposta. Apesar de os exemplos presentes no Capítulo~\ref{cap:posicionamento} usarem como pontos de interesse hospitais, corpos de bombeiros e departamentos de polícia, a abordagem proposta permite a inserção de qualquer instalação como ponto de interesse relevante para a classificação, definindo ainda pesos diferentes de acordo com a sua importância para o evento em questão. Com essa flexibilidade, a classificação de zonas de mitigação proposta pode ser usada em qualquer cidade e com qualquer tipo de evento, desde que o interessado defina apropriadamente os pontos de interesses relevantes para o processo de mitigação do evento em questão e seus pesos.

Como forma de demonstrar a aplicabilidade da abordagem de classificação proposta, o Capítulo~\ref{cap:posicionamento} também apresenta algoritmos para posicionamento de unidades de sensoriamento a partir do resultado da classificação. Nos posicionamentos propostos, é dada uma maior prioridade a áreas com menores níveis de mitigação, ou seja, áreas nas quais o início do processo de mitigação tende a ser mais demorado devido à sua distância dos centros de resposta. Nesta aplicação, o resultado da classificação dos níveis de mitigação foram utilizados como suporte para a elaboração de uma abordagem de detecção de eventos. Outras aplicações podem ser desenvolvidas ou melhoradas a partir da classificação proposta, como por exemplo, planejamento de instalação de novos centros de resposta, avaliação de riscos pela defesa civil, etc.

De forma a permitir outros pesquisadores a utilizarem e experimentarem as soluções elaboradas nesta pesquisa, uma ferramenta \emph{web} foi desenvolvida e publicada, o CityZones. O CityZones pode ser usado por qualquer um, requerendo apenas um cadastro rápido e gratuito na própria ferramenta. Fazendo uso de dados públicos a bertos, ela permite que um usuário possa demarcar uma área de interesse para ser classificada e definir os parâmetros da classificação, como quais pontos de interesse utilizar e seus pesos. A ferramenta permite também que um dos algoritmos de posicionamentos propostos nesta tese seja aplicado a partir do resultado da classificação. A expectativa é de que o CityZones possa ser utilizado por outros pesquisadores como suporte para o desenvolvimento de novas abordagens e também para avaliação de riscos em cenários reais, de forma que este projeto de doutorado tenha então disponibilizado uma ferramenta útil para o desenvolvimento das cidades inteligentes.

Expandindo a abordagem de classificação deste projeto, o Capítulo~\ref{cap:aom} acrescentou os conceitos de área de mitigação, distância extendida e área de interesse poligonal. Sabendo que o algorítmo inicial considerava apenas centros de resposta dentro da área de interesse, o conceito de distância extendida tem como objetivo extender a área considerada dos pontos de interesse para além das fronteiras da área de interesse que está sendo classificada. Desta forma, cria-se uma área de mitigação na qual centros de resposta fora da área de interesse, porém ainda dentro da área de mitigação, possam ser considerados. Isto resolveu um problema recorrente de não considerar instalações que estavam fora da área de interesse, mas bem próximo das zonas classificadas. Além disto, esta etapa da pesquisa melhorou o algoritmo de posicionamento \emph{Restricted+}, criando um novo algoritmo chamado \emph{Enhanced Positioning} que considera as posições das ruas e avenidas de uma região para o posicionamento das unidades de sensoriamento, alcançando assim uma situação mais realista na qual os gestores de uma cidade estão limitados quanto às localidades em que sensores poderão ser instalados.

Por fim, a pesquisa executada adicionou mais uma camada ao algoritmo de classificação de risco voltada a eventos de inundação. Esta nova camada corrobora com o objetivo de produzir uma abordagem generalista mesmo com uma camada específica, pois ela demonstra que a equação final de classificação pode receber novos termos correspondentes a novas camadas com objetivos distintos, produzindo resultados que podem ser utilizados também para fins de posicionamento de sensores, planejamentos urbanos, etc. Esta nova camada, apresentada no Capítulo~\ref{cap:flood}, novamente faz uso de dados públicos e abertos de elevação e inclinação de terreno, além da localização de rios, de forma a computar o risco de inundação das zonas em uma área de interesse. Associado com a presença de centros de resposta, essa nova camada adiciona o cômputo do risco de inundação de cada zona.

O uso de modelagem urbana e dados georeferenciados nesta pesquisa permitiu que os objetivos propostos fossem alcançados. Em se tratando de soluções e abordagens para o gerenciamento de emergências em cidades inteligentes, modelos urbanos se fazem totalmente imprescindíveis já que eles permitirão o projeto e simulação das abordagens considerando os parâmetros do objeto de pesquisa real. Associado a eles, dados georeferenciados permitem o desenvolvimento de abordagens que requerem a localização de pontos específicos no modelo, como os centros de resposta e coordenadas das zonas, utilizados em todo o momento nas abordagens propostas nesta pesquisa.

Pensando em trabalhos de pesquisa futuros, foi elaborado uma \emph{survey} sobre abordagens envolvendo dados georeferenciados, ferramentas GIS, sensoriamento remoto e fontes de dados públicos. O resultado desta revisão irá embasar a continuidade da pesquisa em gerenciamento de emergências, permitindo que novas fontes de dados possam ser adicionadas às abordagens já desenvolvidas, expandindo-as e fazendo com que elas se tornem ainda mais completas e relevantes para sua aplicação em cenários de uso reais.

Por fim, nenhum grande centro urbano está livre de eventos de emergência. Sejam eles quais forem, de causas naturais ou antropogênicas, tais eventos irão ocorrer em um dado momento. Portanto, é importante que os grandes centros urbanos estejam preparados para lidar com estas situações, reforçando a necessidade de pesquisa e desenvolvimento em abordagens de detecção, alerta, mitigação e planejamento de gerenciamento de emergências. Neste sentido, as produções deste projeto de doutorado são relevantes já que fornecem abordagens para classificação de zonas e ferramentas para sua aplicabilidade, permitindo que gestores e autoridades públicas possam planejar a gerência de emergências nas cidades inteligentes.
