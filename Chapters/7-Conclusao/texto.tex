% --------------------------------------------------
% Capítulo 7: Considerações finais
% --------------------------------------------------
\chapter{Conclusions}\label{cap:conclusao}
\selectlanguage{english}

% Gerenciamento de emergências é um tópico bastante relevante e de grande interesse por gestores e autoridades de grandes centros urbanos. Conforme apresentado no Capítulo~\ref{cap:survey}, diversos pesquisadores trabalham neste assunto com o objetivo de desenvolver abordagens e soluções para detecção, alerta e mitigação de eventos de emergência em cidades inteligentes, fazendo uso de tecnologias da informação e comunicação em geral. Porém, a maioria das soluções propostas atacam um cenário específico, seja um evento de emergência ou uma cidade em particular.

Emergency management is a very relevant topic and one of great interest to managers and authorities in large urban centres. As presented in Chapter~\ref{cap:survey}, several researchers are working on this subject with the goal of developing approaches and solutions for detecting, alerting and mitigating emergency events in smart cities, making use of information and communication technologies in general. However, most of the proposed solutions tackle a specific scenario, be it an emergency event or a particular city.

% Esta pesquisa de doutorado propôs uma abordagem generalista, que não considera nenhum evento de emergência específico nem uma região em particular. Usando como ponto de partida a localização de centros de resposta a emergências, a abordagem de classificação de risco proposta no Capítulo~\ref{cap:posicionamento} é capaz de segmentar a área de interesse a qual se deseja classificar em zonas de mitigação, definindo então os níveis de mitigação de cada zona de acordo com a presença e distância de centros de resposta. Apesar de os exemplos presentes no Capítulo~\ref{cap:posicionamento} usarem como pontos de interesse hospitais, corpos de bombeiros e departamentos de polícia, a abordagem proposta permite a inserção de qualquer instalação como ponto de interesse relevante para a classificação, definindo ainda pesos diferentes de acordo com a sua importância para o evento em questão. Com essa flexibilidade, a classificação de zonas de mitigação proposta pode ser usada em qualquer cidade e com qualquer tipo de evento, desde que o interessado defina apropriadamente os pontos de interesses relevantes para o processo de mitigação do evento em questão e seus pesos.

This doctoral research has proposed a generalised approach, which does not consider any specific emergency event or particular region. Using the location of emergency response centres as a starting point, the risk assessment approach proposed in Chapter~\ref{cap:posicionamento} is capable of segmenting the area of interest to be classified into mitigation zones, and then defining the mitigation levels of each zone according to the presence and distance of response centres. Although the examples in Chapter~\ref{cap:posicionamento} use hospitals, fire stations and police departments as points of interest, the proposed approach allows any facility to be inserted as a relevant point of interest for assessment, while also defining different weights according to their importance to the event in question. With this flexibility, the proposed mitigation zone assessment can be used in any city and with any type of event, as long as the interested party appropriately defines the relevant points of interest for the mitigation process of the event in question and their weights.

% Como forma de demonstrar a aplicabilidade da abordagem de classificação proposta, o Capítulo~\ref{cap:posicionamento} também apresenta algoritmos para posicionamento de unidades de sensoriamento a partir do resultado da classificação. Nos posicionamentos propostos, é dada uma maior prioridade a áreas com menores níveis de mitigação, ou seja, áreas nas quais o início do processo de mitigação tende a ser mais demorado devido à sua distância dos centros de resposta. Nesta aplicação, o resultado da classificação dos níveis de mitigação foram utilizados como suporte para a elaboração de uma abordagem de detecção de eventos. Outras aplicações podem ser desenvolvidas ou melhoradas a partir da classificação proposta, como por exemplo, planejamento de instalação de novos centros de resposta, avaliação de riscos pela defesa civil, etc.

As a way of demonstrating the applicability of the proposed assessment approach, Chapter~\ref{cap:posicionamento} also presents algorithms for positioning sensing units based on the assessment result. In the proposed positioning, greater priority is given to areas with lower levels of mitigation, i.e.\ areas where the starting of the mitigation process tends to take longer due to their distance from the response centres. In this application, the results of the assessment of mitigation levels were used to support the development of an event detection approach. Other applications can be developed or improved on the basis of the proposed assessment, such as planning the installation of new response centres, risk assessment by emergency authorities, etc.

% De forma a permitir outros pesquisadores a utilizarem e experimentarem as soluções elaboradas nesta pesquisa, uma ferramenta \emph{web} foi desenvolvida e publicada, o CityZones. O CityZones pode ser usado por qualquer um, requerendo apenas um cadastro rápido e gratuito na própria ferramenta. Fazendo uso de dados públicos a bertos, ela permite que um usuário possa demarcar uma área de interesse para ser classificada e definir os parâmetros da classificação, como quais pontos de interesse utilizar e seus pesos. A ferramenta permite também que um dos algoritmos de posicionamentos propostos nesta tese seja aplicado a partir do resultado da classificação. A expectativa é de que o CityZones possa ser utilizado por outros pesquisadores como suporte para o desenvolvimento de novas abordagens e também para avaliação de riscos em cenários reais, de forma que este projeto de doutorado tenha então disponibilizado uma ferramenta útil para o desenvolvimento das cidades inteligentes.

In order to allow other researchers to use and experiment with the solutions developed in this research, a web tool has been developed and published: CityZones. CityZones can be used by anyone, requiring only a quick and free registration on the tool itself. Using open public data, it allows a user to delimit an area of interest to be classified and define the parameters of the assessment, such as which points of interest to use and their weights. The tool also allows one of the positioning algorithms proposed in this thesis to be applied based on the assessment result. The hope is that CityZones can be used by other researchers to support the development of new approaches and also to assess risks in real scenarios, so that this doctoral project has provided a useful tool for the development of smart cities.

% Expandindo a abordagem de classificação deste projeto, o Capítulo~\ref{cap:aom} acrescentou os conceitos de área de mitigação, distância extendida e área de interesse poligonal. Sabendo que o algorítmo inicial considerava apenas centros de resposta dentro da área de interesse, o conceito de distância extendida tem como objetivo extender a área considerada dos pontos de interesse para além das fronteiras da área de interesse que está sendo classificada. Desta forma, cria-se uma área de mitigação na qual centros de resposta fora da área de interesse, porém ainda dentro da área de mitigação, possam ser considerados. Isto resolveu um problema recorrente de não considerar instalações que estavam fora da área de interesse, mas bem próximo das zonas classificadas. Além disto, esta etapa da pesquisa melhorou o algoritmo de posicionamento \emph{Restricted+}, criando um novo algoritmo chamado \emph{Enhanced Positioning} que considera as posições das ruas e avenidas de uma região para o posicionamento das unidades de sensoriamento, alcançando assim uma situação mais realista na qual os gestores de uma cidade estão limitados quanto às localidades em que sensores poderão ser instalados.

Expanding on this project's assessment approach, Chapter~\ref{cap:aom} added the concepts of mitigation area, extended distance and polygonal area of interest. Knowing that the initial algorithm only considered response centres within the area of interest, the extended distance concept aims to extend the considered area of points of interest beyond the borders of the area of interest being classified. In this way, a mitigation area is created in which response centres outside the area of interest, but still within the mitigation area, can be considered. This solved a recurring problem of not considering facilities that were outside the area of interest but very close to the classified zones. In addition, this stage of the research improved the Restricted+ positioning algorithm, creating a new algorithm called Enhanced Positioning that takes into account the positions of streets and avenues in a region for positioning the sensing units, thus achieving a more realistic situation in which city managers are limited as to where sensors can be installed.

% Por fim, a pesquisa executada adicionou mais uma camada ao algoritmo de classificação de risco voltada a eventos de inundação. Esta nova camada corrobora com o objetivo de produzir uma abordagem generalista mesmo com uma camada específica, pois ela demonstra que a equação final de classificação pode receber novos termos correspondentes a novas camadas com objetivos distintos, produzindo resultados que podem ser utilizados também para fins de posicionamento de sensores, planejamentos urbanos, etc. Esta nova camada, apresentada no Capítulo~\ref{cap:flood}, novamente faz uso de dados públicos e abertos de elevação e inclinação de terreno, além da localização de rios, de forma a computar o risco de inundação das zonas em uma área de interesse. Associado com a presença de centros de resposta, essa nova camada adiciona o cômputo do risco de inundação de cada zona.

Finally, the research carried out added another layer to the risk assessment algorithm for flood events. This new layer corroborates the objective of producing a generalist approach even with a specific layer, as it demonstrates that the final assessment equation can receive new terms corresponding to new layers with different goals, producing results that can also be used for sensor positioning purposes, urban planning, etc. This new layer, presented in Chapter~\ref{cap:flood}, again makes use of public and open data on terrain elevation and slope, as well as the location of rivers, in order to compute the flood risk of zones in an area of interest. Associated with the presence of response centres, this new layer adds the computation of the flood risk of each zone.

% O uso de modelagem urbana e dados georeferenciados nesta pesquisa permitiu que os objetivos propostos fossem alcançados. Em se tratando de soluções e abordagens para o gerenciamento de emergências em cidades inteligentes, modelos urbanos se fazem totalmente imprescindíveis já que eles permitirão o projeto e simulação das abordagens considerando os parâmetros do objeto de pesquisa real. Associado a eles, dados georeferenciados permitem o desenvolvimento de abordagens que requerem a localização de pontos específicos no modelo, como os centros de resposta e coordenadas das zonas, utilizados em todo o momento nas abordagens propostas nesta pesquisa.

The use of urban modelling and georeferenced data in this research enabled the proposed objectives to be achieved. When it comes to solutions and approaches for emergency management in smart cities, urban modelling is absolutely essential as it will allow approaches to be designed and simulated taking into account the parameters of the real research object. In addition, georeferenced data enables the development of approaches that require the location of specific points in the model, such as response centres and zone coordinates, which are used at all times in the approaches proposed in this research.

% Pensando em trabalhos de pesquisa futuros, foi elaborado uma \emph{survey} sobre abordagens envolvendo dados georeferenciados, ferramentas GIS, sensoriamento remoto e fontes de dados públicos. O resultado desta revisão irá embasar a continuidade da pesquisa em gerenciamento de emergências, permitindo que novas fontes de dados possam ser adicionadas às abordagens já desenvolvidas, expandindo-as e fazendo com que elas se tornem ainda mais completas e relevantes para sua aplicação em cenários de uso reais.

With future research in mind, a survey was carried out on approaches involving georeferenced data, GIS tools, remote sensing and public data sources. The results of this review will inform further research into emergency management, allowing new data sources to be added to the approaches already developed, expanding them and making them even more complete and relevant for application in real-life scenarios.

% Por fim, nenhum grande centro urbano está livre de eventos de emergência. Sejam eles quais forem, de causas naturais ou antropogênicas, tais eventos irão ocorrer em um dado momento. Portanto, é importante que os grandes centros urbanos estejam preparados para lidar com estas situações, reforçando a necessidade de pesquisa e desenvolvimento em abordagens de detecção, alerta, mitigação e planejamento de gerenciamento de emergências. Neste sentido, as produções deste projeto de doutorado são relevantes já que fornecem abordagens para classificação de zonas e ferramentas para sua aplicabilidade, permitindo que gestores e autoridades públicas possam planejar a gerência de emergências nas cidades inteligentes.

Finally, no major urban centre is free from emergency events. Whatever they may be, from natural or anthropogenic causes, such events will occur at some point. It is therefore important for large urban centres to be prepared to deal with these situations, reinforcing the need for research and development into approaches to emergency detection, warning, mitigation and management planning. In this sense, the outputs of this doctoral project are relevant as they provide approaches for classifying zones and tools for their applicability, allowing managers and public authorities to plan emergency management in smart cities.
