% --------------------------------------------------
% Capítulo 1: Introdução
% --------------------------------------------------
\chapter{Introdução}\label{cap:introducao}

\begin{refsection}

As cidades estão se tornando maiores e mais complexas a cada ano. Atualmente, a disponibilidade de serviços e facilidades oferecidos pelos grandes centros urbanos, aliados ao avanço da tecnologia, têm feito várias melhorias na qualidade de vida dos cidadãos nestes locais, atraindo mais pessoas para as cidades. Na verdade, é esperado que 68~\% da população mundial esteja vivendo em áreas urbanas até 2050~\cite{de_Amorim_2019}.

Este aumento da densidade demográfica trás novos desafios para as zonas urbanas. Em uma grande cidade, não é raro acontecer acidentes que vão desde pequenas colisões de veículos, passando por incêndios e enchentes, até a quedas de grandes edifícios. Anteriormente, pequenos incidentes apenas requeriam que os socorristas pudessem chegar até a área de crise para resgatar vítimas e mitigar a situação. À medida que as cidades crescem em área construída, habitantes e infraestrutura, estes pequenos incidentes podem trazer grandes efeitos colaterais, já que se torna mais difícil alcançar algumas localidades específicas devido ao tráfego pesado de veículos, além de que a alta densidade urbana pode propagar o evento mais rapidamente, como por exemplo, um incêndio em um prédio que se espalha pela vizinhança~\cite{Ghosh201713}.

Adicionalmente ao crescimento das áreas urbanas, mudanças climáticas estão aumentando a ocorrência e intensidade de desastres naturais, consequentemente aumentando o número de situações de emergência que uma cidade precisa enfrentar~\cite{boukerche2018smart}. Desastres naturais como terremotos, inundações, tsunamis e outros, que são comuns em alguns locais, podem causar perdas e danos severos que são altamente amplificados pela densidade urbana~\cite{Gosavi2020582}. Tais eventos exigem uma resposta rápida para mitigação de crises, alertando cidadãos e autoridades sobre os eventos que ocorrem, evitando mais perdas, tanto de vidas como econômicas~\cite{Costa_2019}.

Recentemente, iniciativas de cidades inteligentes têm sido propostas e implementadas, aliadas às novas tecnologias, para melhorar a qualidade de vida do cidadão, tornando a vivência urbana mais fácil e trazendo um melhor gerenciamento de seus recursos~\cite{Alkhatib2019771}. Estas iniciativas tecnológicas e inteligentes são combinadas com necessidades emergenciais atuais para criar soluções viáveis que trazem mais segurança contra desastres naturais e antropogênicos para os cidadãos de uma cidade inteligente.

%Roughly speaking, knowing the best possible approaches to face emergencies in a high-density urban area, as well as mapping the available smart cities initiatives, is expected to be of great help to support new developments in research areas. Combined with wireless sensor networks, ubiquitous connectivity and Internet of Things (IoT) infrastructure, several works have aimed to manage emergency situations in a smart city scenario~\cite{Seo2017}. Although such issues are already addressed by many researchers, the proposed approaches tackle specifics parts of an emergency or specific types of situation~\cite{Soyata2019}. Therefore, there is a need for a comprehensive solution that can tackle as many types of emergency as possible, addressing the whole situation, from detection to alerting and mitigation.

A redução nos custos de componentes eletrônicos para construção de sensores e atuadores colaboram com a redução da complexidade de implantação destes sistemas. Com os avanços nas pesquisas nas linhas de Cidades Inteligentes, Internet das Coisas e Redes de Sensores Sem Fios, aliados ao contínuo crescimento dos centros urbanos, a demanda pelo uso destas tecnologias vêm crescendo cada vez mais~\cite{8472005}. Este crescimento contínuo dos centros urbanos, aumentando o fluxo de pessoas e veículos, com o crescimento de serviços, grandes construções, maior demanda energética, dentre outros aspectos associados ao crescimento de uma grande metrópole, exige uma resposta cada vez mais rápida a certas demandas de uma Cidade Inteligente. Uma dessas demandas é a rápida resposta a emergências e crises.

Para que seja possível a execução de respostas rápidas às emergências, novas tecnologias podem ser empregadas para alterar o paradigma atual de gerenciamento de crises em áreas urbanas, visando o aumento da eficiência e minimização de vítimas e perdas econômicas. De fato, tais tecnologias podem ser empregadas para monitoramento de diversas variáveis que compõem o complexo sistema que rege a vida nas cidades, como a qualidade do ar~\cite{Silva_2019}, o tráfego e a mobilidade urbana~\cite{Calderoni_2014}, a temperatura ambiental~\cite{silva2020climifba}, os níveis de chuva~\cite{Xu_2016}, os incidentes de segurança~\cite{Alkhatib2019771}, acidentes~\cite{Thompson_2010}, dentre muitas outras. Adicionalmente, informações acerca da distribuição socioeconômica dos habitantes, bem como dados geográficos das zonas urbanas, também são fundamentais e complementam a percepção das cidades em relação a um gerenciamento de crise mais eficiente~\cite{KONTOKOSTA2018272}. Combinando essas informações obtidas de diversas fontes, como base de dados distribuídas e de acesso aberto, pretende-se classificar zonas de risco que possam permitir às autoridades planejar o gerencimaento de emergências nas cidades, possibilitando respostas rápidas às crises nos ambientes urbanos. %abrangendo desde a comunicação às autoridades responsáveis até a execução de ações diretas como alteração de tráfego e alocação de veículos de emergência para zonas afetadas.

Neste cenário, novas tecnologias de sensoriamento e processamento de dados são de grande auxílio no desenvolvimento de soluções para a detecção de emergências e gerenciamento de crises, porém, desde que as diversas variáveis relacionadas ao modo de vida urbano sejam propriamente consideradas. A modelagem abrangente desse cenário é, portanto, parte fundamental para o desenvolvimento de soluções focadas nos grandes desafios do gerenciamento de crises nas cidades.

% --------------------------------------------------
% Objetivos
% --------------------------------------------------
\section{Objetivos}\label{sec:objetivos}

Algumas soluções já propostas para gerenciamento de emergências em cidades inteligentes focam em eventos e locais específicos, porém, uma abordagem generalista é necessária para que uma cidade inteligente seja totalmente atendida quanto à variedade de emergências, começando pela classificação de risco das áreas monitoradas. %Além disso, as soluções existentes focam em monitoramento, alerta ou mitigação em situações específicas, havendo uma necessidade de um conjunto de soluções que abrange os três estados de uma emergência para diversos tipos de situações.

Partindo destas premissas, esta tese de doutorado tem como objetivo produzir um conjunto de soluções computacionais, algoritmos e processos para classificação de zonas de risco a emergências em zonas urbanas, fornecendo assim uma metodologia generalista, dando suporte ao desenvolvimento do conceito de cidades inteligentes.

%As hipóteses para o desenvolvimento deste projeto são que a junção de diversos sistemas de monitoramento e sensoriamento especializados podem tornar uma cidade inteligente capaz de detectar, em tempo hábil, qualquer tipo de emergência e que o desenvolvimento de uma solução completa para gerenciamento de emergências em cidades inteligentes pode aumentar a eficiência dos processos de detecção e respostas às crises. Para isto, pretende-se:

As hipóteses para o desenvolvimento deste projeto são que a devida classificação de riscos em uma área de interesse pode permitir um correto planejamento e desenvolvimento de ações de detecção, alerta e mitigação de eventos de emergência em uma cidade inteligente. Para isto, pretende-se:

\begin{itemize}
  %\item Identificar as variáveis que melhor representam o comportamento das cidades em situações de crise, relacionando tanto os fatores de risco (que provocam uma emergência) como os elementos associados às operações de alerta e resgate (o que fazer depois que uma emergência está ocorrendo);
  %\item Desenvolver soluções que possibilitem o monitoramento, direto e indireto, das variáveis que indiquem a ocorrência de eventos críticos em uma cidade inteligente. Tal monitoramento pode ser realizado utilizando sensores, \textit{smartphones}, bases de dados públicas ou qualquer fonte de dados que possam ajudar a modelar o comportamento das cidades em relação a ocorrência de emergências;
  %\item Elaborar mecanismos de alerta de emergências, conectando sistemas de detecção e mitigação, de maneira resiliente, considerando eventuais problemas de infraestrutura de rede e comunicação no cenário pós-crise;
  %\item Propor algoritmos e procedimentos para mitigação de crises, melhorando respostas a eventos, como por exemplo, redirecionando o tráfego através de semáforos inteligentes para liberação de pistas para veículos de emergência, criação e orientação de corredores de fuga, desligamento automático de fontes de energia que possam causar incêndios devido à ocorrência de uma emergência, dentre outras ações;
  %\item Criar modelos matemáticos relacionados à ocorrência de emergências nas cidades, permitindo a simulação de crises e a verificação da qualidade dos serviços de respostas nas cidades. Tais modelos serão validados de acordo com dados reais obtidos de cidades analisadas;
  %\item Proposição de métricas e indicadores de risco que possam ser adotados nas cidades, guiando os agentes públicos acerca das melhorias que possam ser feitas para aperfeiçoar as respostas às emergências.
  \item Identificar as variáveis que influenciam a avaliação de risco em eventos de crise em uma cidade, permitindo a execução de processos de mitigação que possam trazer a região à sua situação pré-crise;
  \item Desenvolver um modelo matemático generalista capaz de computar a classificação de risco em zonas segmentadas de uma área de interesse, permitindo a adição de camadas para tipos diferentes de eventos;
  \item Desenvolver uma ferramenta de acesso aberto como prova de conceito, capaz de demonstrar a aplicação da abordagem de classificação de risco proposta;
  \item Desenvolver algoritmos para posicionamento de unidades de sensoriamento baseado no resultado da classificação de riscos a fim de demonstrar a aplicabilidade da proposta.
\end{itemize}

% --------------------------------------------------
% Contribuições
% --------------------------------------------------
\section{Contribuições}\label{sec:contribuicoes}

Ao propor uma abordagem generalista de classificação de riscos em cidades inteligentes, esta tese de doutorado contribui para alcançar as metas 11.b e 11.5 do ODS 11 (Objetivos de Desenvolvimento Sustentável), a qual trata de cidades e comunidades sustentáveis~\cite{ipeaods}. As metas mencionadas são:

\begin{itemize}
  \item 11.b: ``Até 2030, aumentar significativamente o número de cidades que possuem políticas e planos desenvolvidos e implementados para mitigação, adaptação e resiliência a mudanças climáticas e gestão integrada de riscos de desastres de acordo com o Marco de SENDAI''
  \item 11.5: ``Até 2030, reduzir significativamente o número de mortes e o número de pessoas afetadas por desastres naturais de origem hidrometeorológica e climatológica, bem como diminuir substancialmente o número de pessoas residentes em áreas de risco e as perdas econômicas diretas causadas por esses desastres em relação ao produto interno bruto, com especial atenção na proteção de pessoas de baixa renda e em situação de vulnerabilidade''
\end{itemize}

% Indiretamente, esta tese também poderá contribuir com as metas 3.6 e 3.9 do ODS 3, a qual trata de saúde e bem-estar:

% \begin{itemize}
%   \item 3.6: ``Até 2030, reduzir pela metade as mortes e lesões por acidentes no trânsito''
%   \item 3.9: ``Até 2030, reduzir substancialmente o número de mortes e doenças por produtos químicos perigosos, contaminação e poluição do ar e água do solo''
% \end{itemize}

% --------------------------------------------------
% Publicações
% --------------------------------------------------
\section{Publicações}\label{sec:publicacoes}

Durante a pesquisa e o desenvolvimento desta tese os seguintes artigos foram publicados. O primeiro artigo foi uma revisão narrativa sobre o estado da arte no gerenciamento de emergências em cidades inteligentes, sendo o alicerce para construção desse trabalho de Doutorado. Em sequência, cada etapa concluída da pesquisa produziu um novo artigo que foi publicado em congressos e revistas, definindo marcos na evolução do trabalho. Por fim, vale salientar que alguns desses artigos foram publicados em revistas científicas internacionais de alta relevância, indexadas nas principais bases internacionais, com classificação Qualis A1, A2 e A3.

\begin{itemize}
  \item COSTA, Daniel G. et al. A Survey of Emergencies Management Systems in Smart Cities.\ \textbf{IEEE Access}, 2022. DOI:\ \hyperref{https://doi.org/10.1109/access.2022.3180033}{}{}{10.1109/access.2022.3180033}. Qualis A1.
  \item PEIXOTO, João Paulo J. et al. Optimizing the deployment of multi-sensors emergencies detection units based on the presence of response centers in smart cities.\ \textbf{IEEE International Smart Cities Conference 2022, Cyprus}, 2022.\\DOI:\ \hyperref{https://doi.org/10.1109/ISC255366.2022.9922075}{}{}{10.1109/ISC255366.2022.9922075}
  \item PEIXOTO, João Paulo J. et al. On the Positioning of Emergencies Detection Units Based on Geospatial Data of Urban Response Centres.\ \textbf{Sustainable Cities and Society}, 2023. DOI:\ \hyperref{https://doi.org/10.1016/j.scs.2023.104713}{}{}{10.1016/j.scs.2023.104713}. Qualis A1.
  \item PEIXOTO, João Paulo J. et al. CityZones: A geospatial multi-tier software tool to compute urban risk zones.\ \textbf{SoftwareX}, 2023. DOI:\ \hyperref{https://doi.org/10.1016/j.softx.2023.101409}{}{}{10.1016/j.softx.2023.101409}. Qualis A3.
  \item PEIXOTO, João Paulo J. et al. Enhancing the computation of risk zones based on emergency-related infrastructure in smart cities.\ \textbf{IEEE International Smart Cities Conference 2023, Romania}, 2023. DOI:\ \hyperref{https://doi.org/10.1109/ISC257844.2023.10293416}{}{}{10.1109/ISC257844.2023.10293416}
  \item PEIXOTO, João Paulo J. et al. A geospatial dataset of urban infrastructure for emergency response in Portugal.\ \textbf{Data in brief}, v. 50, p. 109593, 2023. DOI:\ \hyperref{https://doi.org/10.1016/j.dib.2023.109593}{}{}{10.1016/j.dib.2023.109593}. Qualis A3.
  \item PEIXOTO, João Paulo J. et al. Exploiting geospatial data of connectivity and urban infrastructure for efficient positioning of emergency detection units in smart cities.\ \textbf{Computers, Environment and Urban Systems}, v. 107, p. 102054, 2024. DOI:\ \hyperref{https://doi.org/10.1016/j.compenvurbsys.2023.102054}{}{}{10.1016/j.compenvurbsys.2023.102054}. Qualis A1.
  \item PEIXOTO, João Paulo J.; COSTA, Daniel G.; ROCHA, Washington de J. S. da Franca. Analisando Centros de Resposta à Emergências em Capitais Brasileiras Utilizando uma Abordagem Baseada em Dados Geoespaciais Colaborativos.\ \textbf{XIX Simpósio Brasileiro de Sistemas Colaborativos}. (ACEITO PARA PUBLICAÇÃO)
  \item PEIXOTO, João Paulo J. et al. Towards Flood-resilient Smart Cities: A Data-driven Risk Assessment Approach Based on Geographical Risks and Emergency Response Infrastructure.\ \textbf{MDPI Smart Cities}. Qualis A1. (ACEITO PARA PUBLICAÇÃO)
  \item COSTA, Daniel G. et al. Achieving sustainable smart cities through geospatial data-driven approaches.\ \textbf{MDPI Sustainability}, 2024. DOI:\ \hyperref{https://doi.org/10.3390/su16020640}{}{}{10.3390/su16020640}. Qualis A2.
\end{itemize}

% --------------------------------------------------
% Estrutura da tese
% --------------------------------------------------
\section{Estrutura da Tese}\label{sec:estrutura}

Esta tese apresenta o progresso do projeto de pesquisa desenvolvido, que culminou em uma abordagem generalista para classificação de riscos em cidades inteligentes. Nos próximos capítulos estão os artigos com as maiores contribuições para o desenvolvimento da pesquisa. No Capítulo~\ref{cap:survey}, uma revisão bibliográfica do tipo narrativa apresenta o estado da arte quanto às abordagens de detecção, alerta, mitigação e gerenciamento de emergências em cidades inteligentes, parte inicial e imprescindível para um projeto de pesquisa. O Capítulo~\ref{cap:posicionamento} apresenta o desenvolvimento incial da abordagem para classificação de risco, como o conceito de zonas de risco, área de interesse e pontos de interesse, além de alguns algoritmos de posicionamento de unidades de sensoriamento, demonstrando a aplicabilidade da abordagem proposta. O Capítulo~\ref{cap:cityzones} apresenta o CityZones, uma ferramenta \emph{web} desenvolvida para permitir o uso e aplicação da classificação de risco por qualquer pesquisador. O Capítulo~\ref{cap:aom} apresenta o desenvolvimento do conceito de Área de Mitigação, amplicando a área de atuação dos Pontos de Interesse no processo de classificação de risco, além de aprimorar o algoritmo de posicionamento de unidades de sensoriamento. O Capítulo~\ref{cap:flood} extende a abordagem proposta, demonstrando a possibilidade de adição de camadas de classificação, aplicando uma camada para classificação de risco a inundações.

\printbibliography[heading=subbibliography]

\end{refsection}
