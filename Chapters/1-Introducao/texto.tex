% --------------------------------------------------
% Capítulo 1: Introdução
% --------------------------------------------------
\selectlanguage{english}
\chapter{Introduction}\label{cap:introducao}

\begin{refsection}

% As cidades estão se tornando maiores e mais complexas a cada ano. Atualmente, a disponibilidade de serviços e facilidades oferecidos pelos grandes centros urbanos, aliados ao avanço da tecnologia, têm feito várias melhorias na qualidade de vida dos cidadãos nestes locais, atraindo mais pessoas para as cidades. Na verdade, é esperado que 68~\% da população mundial esteja vivendo em áreas urbanas até 2050~\cite{de_Amorim_2019}.

Cities are becoming larger and more complex every year. Currently, the availability of services and facilities offered by large urban centres, coupled with advances in technology, have made several improvements to the quality of life of citizens in these places, attracting more people to cities. In fact, it is expected that 68~\% of the world's population will be living in urban areas by 2050~\cite{de_Amorim_2019}.

% Este aumento da densidade demográfica trás novos desafios para as zonas urbanas. Em uma grande cidade, não é raro acontecer acidentes que vão desde pequenas colisões de veículos, passando por incêndios e enchentes, até a quedas de grandes edifícios. Anteriormente, pequenos incidentes apenas requeriam que os socorristas pudessem chegar até a área de crise para resgatar vítimas e mitigar a situação. À medida que as cidades crescem em área construída, habitantes e infraestrutura, estes pequenos incidentes podem trazer grandes efeitos colaterais, já que se torna mais difícil alcançar algumas localidades específicas devido ao tráfego pesado de veículos, além de que a alta densidade urbana pode propagar o evento mais rapidamente, como por exemplo, um incêndio em um prédio que se espalha pela vizinhança~\cite{Ghosh201713}.

This increase in population density brings new challenges for urban areas. In a large city, it is not uncommon for accidents to occur, ranging from minor vehicle collisions, through fires and floods, to the collapse of large buildings. Previously, small incidents only required first responders to reach the crisis area to rescue victims and mitigate the situation. As cities grow in built-up area, inhabitants and infrastructure, these small incidents can have major side effects, as it becomes more difficult to reach specific locations due to heavy vehicle traffic, and high urban density can spread the event more quickly, such as a fire in a building that spreads through the neighbourhood~\cite{Ghosh201713}.

% Adicionalmente ao crescimento das áreas urbanas, mudanças climáticas estão aumentando a ocorrência e intensidade de desastres naturais, consequentemente aumentando o número de situações de emergência que uma cidade precisa enfrentar~\cite{boukerche2018smart}. Desastres naturais como terremotos, inundações, tsunamis e outros, que são comuns em alguns locais, podem causar perdas e danos severos que são altamente amplificados pela densidade urbana~\cite{Gosavi2020582}. Tais eventos exigem uma resposta rápida para mitigação de crises, alertando cidadãos e autoridades sobre os eventos que ocorrem, evitando mais perdas, tanto de vidas como econômicas~\cite{Costa_2019}.

In addition to the growth of urban areas, climate change is increasing the occurrence and intensity of natural disasters, consequently increasing the number of emergency situations a city has to face~\cite{boukerche2018smart}. Natural disasters such as earthquakes, floods, tsunamis and others, which are common in some places, can cause severe losses and damage that are highly amplified by urban density~\cite{Gosavi2020582}. Such events require a rapid response for crisis mitigation, alerting citizens and authorities to the events that occur, preventing further loss of both life and economic~\cite{Costa_2019}.

% Recentemente, iniciativas de cidades inteligentes têm sido propostas e implementadas, aliadas às novas tecnologias, para melhorar a qualidade de vida do cidadão, tornando a vivência urbana mais fácil e trazendo um melhor gerenciamento de seus recursos~\cite{Alkhatib2019771}. Estas iniciativas tecnológicas e inteligentes são combinadas com necessidades emergenciais atuais para criar soluções viáveis que trazem mais segurança contra desastres naturais e antropogênicos para os cidadãos de uma cidade inteligente.

Recently, smart city initiatives have been proposed and implemented, combined with new technologies, to improve citizens' quality of life, making urban living easier and bringing better management of their resources~\cite{Alkhatib2019771}. These technological and intelligent initiatives are combined with current emergency needs to create viable solutions that bring greater security against natural and man-made disasters to the citizens of a smart city.

% A redução nos custos de componentes eletrônicos para construção de sensores e atuadores colaboram com a redução da complexidade de implantação destes sistemas. Com os avanços nas pesquisas nas linhas de Cidades Inteligentes, Internet das Coisas e Redes de Sensores Sem Fios, aliados ao contínuo crescimento dos centros urbanos, a demanda pelo uso destas tecnologias vêm crescendo cada vez mais~\cite{8472005}. Este crescimento contínuo dos centros urbanos, aumentando o fluxo de pessoas e veículos, com o crescimento de serviços, grandes construções, maior demanda energética, dentre outros aspectos associados ao crescimento de uma grande metrópole, exige uma resposta cada vez mais rápida a certas demandas de uma Cidade Inteligente. Uma dessas demandas é a rápida resposta a emergências e crises.

The reduction in the cost of electronic components for building sensors and actuators helps to reduce the complexity of implementing these systems. With the advances in research into smart cities, the Internet of Things and Wireless Sensor Networks, combined with the continuous growth of urban centres, the demand for the use of these technologies is growing more and more~\cite{8472005}.

This continuous growth of urban centres, increasing the flow of people and vehicles, with the growth of services, large buildings, greater energy demand, among other aspects associated with the growth of a large metropolis, requires an increasingly rapid response to certain demands of a smart city. One of these demands is the rapid response to emergencies and crises. In fact, every city is prone to emergency situations, making it necessary to build proper detection and mitigation plans~\cite{Gosavi2020582}. Approaches for emergency detection, alerting, mitigation and risk assessment are very helpful to aid public authorities and stakeholders to prospect initiatives and actions for building resilient cities~\cite{bodoque2016improvement}. In other words, emergencies are going to happen and a resilient city must be prepared with proper detection or mitigation. Both ways, a risk assessment of zones in the city is essential.

% Para que seja possível a execução de respostas rápidas às emergências, novas tecnologias podem ser empregadas para alterar o paradigma atual de gerenciamento de crises em áreas urbanas, visando o aumento da eficiência e minimização de vítimas e perdas econômicas. De fato, tais tecnologias podem ser empregadas para monitoramento de diversas variáveis que compõem o complexo sistema que rege a vida nas cidades, como a qualidade do ar~\cite{Silva_2019}, o tráfego e a mobilidade urbana~\cite{Calderoni_2014}, a temperatura ambiental~\cite{silva2020climifba}, os níveis de chuva~\cite{Xu_2016}, os incidentes de segurança~\cite{Alkhatib2019771}, acidentes~\cite{Thompson_2010}, dentre muitas outras. Adicionalmente, informações acerca da distribuição socioeconômica dos habitantes, bem como dados geográficos das zonas urbanas, também são fundamentais e complementam a percepção das cidades em relação a um gerenciamento de crise mais eficiente~\cite{KONTOKOSTA2018272}. Combinando essas informações obtidas de diversas fontes, como base de dados distribuídas e de acesso aberto, pretende-se classificar zonas de risco que possam permitir às autoridades planejar o gerencimaento de emergências nas cidades, possibilitando respostas rápidas às crises nos ambientes urbanos.

In order to be able to carry out rapid responses to emergencies, new technologies can be employed to change the current paradigm of crisis management in urban areas, with a view to increasing efficiency and minimising casualties and economic losses. In fact, such technologies can be used to monitor various variables that make up the complex system that governs life in cities, such as air quality~\cite{Silva_2019}, traffic and urban mobility~\cite{Calderoni_2014}, ambient temperature~\cite{silva2020climifba}, rainfall levels~\cite{Xu_2016}, safety incidents~\cite{Alkhatib2019771}, accidents~\cite{Thompson_2010}, among many others. In addition, information on the socio-economic distribution of inhabitants, as well as geographical data on urban areas, is also fundamental and complements the perception of cities with regard to more efficient crisis management~\cite{KONTOKOSTA2018272}. By combining this information obtained from various sources, such as distributed and open-access databases, the goal is to classify risk zones that can enable authorities to plan emergency management in cities, enabling rapid responses to crises in urban environments.

% Neste cenário, novas tecnologias de sensoriamento e processamento de dados são de grande auxílio no desenvolvimento de soluções para a detecção de emergências e gerenciamento de crises, porém, desde que as diversas variáveis relacionadas ao modo de vida urbano sejam propriamente consideradas. A modelagem abrangente desse cenário é, portanto, parte fundamental para o desenvolvimento de soluções focadas nos grandes desafios do gerenciamento de crises nas cidades.

In this scenario, new sensing and data processing technologies are of great help in developing solutions for emergency detection and crisis management, provided, however, that the various variables related to the urban way of life are properly considered. Comprehensive modelling of this scenario is therefore a fundamental part of developing solutions focused on the major challenges of crisis management in cities.

% --------------------------------------------------
% Objetivos
% --------------------------------------------------
\section{Goals}\label{sec:objetivos}

% Algumas soluções já propostas para gerenciamento de emergências em cidades inteligentes focam em eventos e locais específicos, porém, uma abordagem generalista é necessária para que uma cidade inteligente seja totalmente atendida quanto à variedade de emergências, começando pela classificação de risco das áreas monitoradas. 

Some solutions already proposed for emergency management in smart cities focus on specific events and locations, but a generalist approach is necessary for a smart city to be fully catered for in terms of the variety of emergencies, starting with the risk classification of the monitored areas. 

% Partindo destas premissas, esta tese de doutorado tem como objetivo produzir um conjunto de soluções computacionais, algoritmos e processos para classificação de zonas de risco a emergências em zonas urbanas, fornecendo assim uma metodologia generalista, dando suporte ao desenvolvimento do conceito de cidades inteligentes.

Based on these premises, this doctoral thesis aims to produce a set of computational solutions, algorithms and processes for classifying emergency risk zones in urban areas, thus providing a generalised methodology to support the development of the concept of smart cities.

% As hipóteses para o desenvolvimento deste projeto são que a devida classificação de riscos em uma área de interesse pode permitir um correto planejamento e desenvolvimento de ações de detecção, alerta e mitigação de eventos de emergência em uma cidade inteligente. Para isto, pretende-se:

The hypothesis behind this thesis is that the proper classification of risks in an area of interest can enable the correct planning and development of actions to detect, alert and mitigate emergency events in a smart city. To this end, the aim is to:

% \begin{itemize}
%   \item Identificar as variáveis que influenciam a avaliação de risco em eventos de crise em uma cidade, permitindo a execução de processos de mitigação que possam trazer a região à sua situação pré-crise;
%   \item Desenvolver um modelo matemático generalista capaz de computar a classificação de risco em zonas segmentadas de uma área de interesse, permitindo a adição de camadas para tipos diferentes de eventos;
%   \item Desenvolver uma ferramenta de acesso aberto como prova de conceito, capaz de demonstrar a aplicação da abordagem de classificação de risco proposta;
%   \item Desenvolver algoritmos para posicionamento de unidades de sensoriamento baseado no resultado da classificação de riscos a fim de demonstrar a aplicabilidade da proposta.
% \end{itemize}

\begin{itemize}
  \item Identify the variables that influence risk assessment in crisis events in a city, enabling the implementation of mitigation processes that can bring the region back to its pre-crisis situation;
  \item Develop a generalised mathematical model capable of computing the risk classification in segmented zones of an area of interest, allowing the addition of layers for different types of events;
  \item Develop an open-access tool as a proof of concept, capable of demonstrating the application of the proposed risk classification approach;
  \item Develop algorithms for positioning sensing units based on the result of risk classification in order to demonstrate the applicability of the proposal.
\end{itemize}

% --------------------------------------------------
% Contribuições
% --------------------------------------------------
\section{Contributions}\label{sec:contribuicoes}

% Ao propor uma abordagem generalista de classificação de riscos em cidades inteligentes, esta tese de doutorado contribui para alcançar as metas 11.b e 11.5 do ODS 11 (Objetivos de Desenvolvimento Sustentável), a qual trata de cidades e comunidades sustentáveis~\cite{ipeaods}. As metas mencionadas são:

By proposing a generalised approach to risk classification in smart cities, this doctoral thesis contributes to achieving goals 11.b and 11.5 of SDG 11 (Sustainable Development Goals), which deals with sustainable cities and communities~\cite{ipeaods}. The goals mentioned are:

\begin{itemize}
  \item 11.5: ``By 2030, significantly reduce the number of deaths and the number of people affected and substantially decrease the direct economic losses relative to global gross domestic product caused by disasters, including water-related disasters, with a focus on protecting the poor and people in vulnerable situations''
  \item 11.b: ``By 2020, substantially increase the number of cities and human settlements adopting and implementing integrated policies and plans towards inclusion, resource efficiency, mitigation and adaptation to climate change, resilience to disasters, and develop and implement, in line with the Sendai Framework for Disaster Risk Reduction 2015--2030, holistic disaster risk management at all levels''
\end{itemize}

Also, this doctoral thesis contibutes with the development and publishing of a proof of concept tool called CityZones, that allows any person to experiment the proposed approach through a web interface at \hyperref{https://cityzones.fe.up.pt}{}{}{cityzones.fe.up.pt}. It is expected that this tool helps not only emergency management researchers as also public authorities responsible for emergency planning in cities. With such a risk assessment tool the planning of mitigation plans and the building of new response centres facilities can be leveraged, providing better infrastructure for crisis management.

% --------------------------------------------------
% Publicações
% --------------------------------------------------
\section{Publications}\label{sec:publicacoes}

% Durante a pesquisa e o desenvolvimento desta tese os seguintes artigos foram publicados. O primeiro artigo foi uma revisão narrativa sobre o estado da arte no gerenciamento de emergências em cidades inteligentes, sendo o alicerce para construção desse trabalho de Doutorado. Em sequência, cada etapa concluída da pesquisa produziu um novo artigo que foi publicado em congressos e revistas, definindo marcos na evolução do trabalho. Por fim, vale salientar que alguns desses artigos foram publicados em revistas científicas internacionais de alta relevância, indexadas nas principais bases internacionais, com classificação Qualis A1, A2 e A3.

During the research and development of this thesis, the following articles were published. The first article was a narrative review of the state of the art in emergency management in smart cities, and was the foundation on which this PhD work was built. Subsequently, each completed stage of the research produced a new article that was published in congresses and journals, defining milestones in the evolution of the work. Finally, it is worth pointing out that some of these articles were published in highly relevant international scientific journals, indexed in the main international databases.

\begin{itemize}
  \item \uppercase{Costa}, Daniel G.; \uppercase{Peixoto}, João Paulo J.; \uppercase{Jesus}, Thiago C.; \uppercase{Portugal}, Paulo; \uppercase{Vasques}, Francisco; RANGEL, Elivelton; \uppercase{Peixoto}, Maycon. A Survey of Emergencies Management Systems in Smart Cities.\ \textbf{IEEE Access}, 2022. DOI:\ \hyperref{https://doi.org/10.1109/access.2022.3180033}{}{}{10.1109/access.2022.3180033}: This work performs a comprehensive survey about critical aspects of emergency management in smart cities, comprising current challenges and promising perspectives. It also proposes a new taxonomy centered on the way emergencies are detected and mitigated. Finally, promising trends are envisioned, paving the way for significant developments in this area.

  \item \uppercase{Peixoto}, João Paulo J.; \uppercase{Costa}, Daniel G.; \uppercase{Rocha}, Washington de J. S. da Franca; \uppercase{Portugal}, Paulo; \uppercase{Vasques}, Francisco. Optimizing the deployment of multi-sensors emergencies detection units based on the presence of response centers in smart cities.\ \textbf{IEEE International Smart Cities Conference 2022, Cyprus}, 2022. DOI:\ \hyperref{https://doi.org/10.1109/ISC255366.2022.9922075}{}{}{10.1109/ISC255366.2022.9922075}: This work presents the first results obtained from the initial risk assessment approach. It also demonstrates how the risk assessment proposed can leverage multi-sensors emergencies detection units positioning, considering the prioritization of more risky zones in a city.

  \item \uppercase{Peixoto}, João Paulo J.; \uppercase{Costa}, Daniel G.; \uppercase{Rocha}, Washington de J. S. da Franca; \uppercase{Portugal}, Paulo; \uppercase{Vasques}, Francisco. On the Positioning of Emergencies Detection Units Based on Geospatial Data of Urban Response Centres.\ \textbf{Sustainable Cities and Society}, 2023. DOI:\ \hyperref{https://doi.org/10.1016/j.scs.2023.104713}{}{}{10.1016/j.scs.2023.104713}: This work enhances the risk assessment approach with a better categorization of risk levels, resulting in a more precise zones classification. It also presents a new multi-sensors emergencies detection units positioning algorithm that considers positioning restrictions in a city, making it suitable for real scenario applications.

  \item \uppercase{Peixoto}, João Paulo J.; \uppercase{Costa}, Daniel G.; \uppercase{Rocha}, Washington de J. S. da Franca; \uppercase{Portugal}, Paulo; \uppercase{Vasques}, Francisco. CityZones: A geospatial multi-tier software tool to compute urban risk zones.\ \textbf{SoftwareX}, 2023.\\DOI:\ \hyperref{https://doi.org/10.1016/j.softx.2023.101409}{}{}{10.1016/j.softx.2023.101409}: This work presents CityZones, a proof of concept tool that runs on the web and allows for any stakeholder and researcher to experiment the proposed approach without the need to implement all the presented algorithms. The paper depicts how the tool works to process the assessment tasks of the users using a multi-tier distributed system.

  \item \uppercase{Peixoto}, João Paulo J.; \uppercase{Costa}, Daniel G.; \uppercase{Rocha}, Washington de J. S. da Franca; \uppercase{Portugal}, Paulo; \uppercase{Vasques}, Francisco. Enhancing the computation of risk zones based on emergency-related infrastructure in smart cities.\ \textbf{IEEE International Smart Cities Conference 2023, Romania}, 2023.\\DOI:\ \hyperref{https://doi.org/10.1109/ISC257844.2023.10293416}{}{}{10.1109/ISC257844.2023.10293416}: This work presents another enhancement of the risk assessment approach by adding polygonal areas of interest and the concept of areas of mitigation, allowing the inclusion of response centres that are not inside the area of interest, but near enough to take part in a mitigation process. It also proposes another enhancement of the restricted positioning algorithm by considering the presence of streets in the area of interest.

  \item \uppercase{Peixoto}, João Paulo J.; \uppercase{Costa}, Daniel G.; \uppercase{Portugal}, Paulo; \uppercase{Vasques}, Francisco. A geospatial dataset of urban infrastructure for emergency response in Portugal.\ \textbf{Data in brief}, v. 50, p. 109593, 2023. DOI:\ \hyperref{https://doi.org/10.1016/j.dib.2023.109593}{}{}{10.1016/j.dib.2023.109593}: This work performs a comparison of the presence of response centres in all cities of Portugal. As a final results a dataset of the computed data for the comparison was published, allowing researchers to use such data for emergency management studies in Portugal.

  \item \uppercase{Peixoto}, João Paulo J.; \uppercase{Bittencourt}, João Carlos N.; \uppercase{Jesus}, Thiago C.; \uppercase{Costa}, Daniel G.; \uppercase{Portugal}, Paulo; \uppercase{Vasques}, Francisco. Exploiting geospatial data of connectivity and urban infrastructure for efficient positioning of emergency detection units in smart cities.\ \textbf{Computers, Environment and Urban Systems}, v. 107, p. 102054, 2024. DOI:\ \hyperref{https://doi.org/10.1016/j.compenvurbsys.2023.102054}{}{}{10.1016/j.compenvurbsys.2023.102054}: This work combines the risk assessment approach proposed in this thesis with connectivity data of the city to proposed a multi-sensors emergencies detection units positioning based not only on risk levels as also on available connectivity. Since the detection units need connectivity to send sensed data and alerts, the connectivity of a zone must be considered. As a result, shadow zones can be revealed allowing stakeholders to take action on connectivity improvements.

  \item \uppercase{Peixoto}, João Paulo J.; \uppercase{Costa}, Daniel G.; \uppercase{Rocha}, Washington de J. S. da Franca. Analisando Centros de Resposta à Emergências em Capitais Brasileiras Utilizando uma Abordagem Baseada em Dados Geoespaciais Colaborativos.\ \textbf{XIX Simpósio Brasileiro de Sistemas Colaborativos}, 2024.\\DOI:\ \hyperref{https://doi.org/10.5753/sbsc.2024.237942}{}{}{10.5753/sbsc.2024.237942}: This work counts the number of response centres of every Brazilian capital and computes a ratio of response centres by inhabitants, making a comparison of how well served is the Brazilian capitals. The results depict the inequality of emergency response capabilities of the cities.

  \item \uppercase{Peixoto}, João Paulo J.; \uppercase{Costa}, Daniel G.; \uppercase{Portugal}, Paulo; \uppercase{Vasques}, Francisco. Flood-Resilient Smart Cities: A Data-Driven Risk Assessment Approach Based on Geographical Risks and Emergency Response Infrastructure.\ \textbf{MDPI Smart Cities}, 2024. DOI:\ \hyperref{https://doi.org/10.3390/smartcities7010027}{}{}{10.3390/smartcities7010027}: This work adds a new layer in the risk assessment approach to compute risk levels regarding floods. The new layer retrieves elevation and geo-referenced data of rivers to assess the risk of flooding of zones regarding both heavy rain and river floods. Flood risk is combined with the presence of response centres to produce an oversall risk assessment considering that, in case of a flood, some response centres must operate to mitigate the emergency.

  \item \uppercase{Costa}, Daniel G.; \uppercase{Bittencourt}, João Carlos N.; \uppercase{Oliveira}, Franklin; \uppercase{Peixoto}, João Paulo J.; \uppercase{Jesus}, Thiago C. Achieving sustainable smart cities through geospatial data-driven approaches.\ \textbf{MDPI Sustainability}, 2024.\\DOI:\ \hyperref{https://doi.org/10.3390/su16020640}{}{}{10.3390/su16020640}: This work reviews the urban geo-data processing and geo-referenced approaches for building sustainable cities, depicting its importance in achieving sustainable development goals. This thesis approach is presented in this review as an example of how geo-referenced data can be used to leverage smart cities applications.
\end{itemize}

% --------------------------------------------------
% Estrutura da tese
% --------------------------------------------------
\section{Thesis Structure}\label{sec:estrutura}

% Esta tese apresenta o progresso do projeto de pesquisa desenvolvido, que culminou em uma abordagem generalista para classificação de riscos em cidades inteligentes. Nos próximos capítulos estão os artigos com as maiores contribuições para o desenvolvimento da pesquisa. No Capítulo~\ref{cap:survey}, uma revisão bibliográfica do tipo narrativa apresenta o estado da arte quanto às abordagens de detecção, alerta, mitigação e gerenciamento de emergências em cidades inteligentes, parte inicial e imprescindível para um projeto de pesquisa. O Capítulo~\ref{cap:posicionamento} apresenta o desenvolvimento incial da abordagem para classificação de risco, como o conceito de zonas de risco, área de interesse e pontos de interesse, além de alguns algoritmos de posicionamento de unidades de sensoriamento, demonstrando a aplicabilidade da abordagem proposta. O Capítulo~\ref{cap:cityzones} apresenta o CityZones, uma ferramenta \emph{web} desenvolvida para permitir o uso e aplicação da classificação de risco por qualquer pesquisador. O Capítulo~\ref{cap:aom} apresenta o desenvolvimento do conceito de Área de Mitigação, amplicando a área de atuação dos Pontos de Interesse no processo de classificação de risco, além de aprimorar o algoritmo de posicionamento de unidades de sensoriamento. O Capítulo~\ref{cap:flood} extende a abordagem proposta, demonstrando a possibilidade de adição de camadas de classificação, aplicando uma camada para classificação de risco a inundações.

This thesis presents the progress of the research project that culminated in a generalised approach for classifying risks in smart cities. The following chapters contain the articles with the greatest contributions to the development of the research. In Chapter~\ref{cap:survey}, a narrative literature review presents the state of the art in approaches to detecting, alerting, mitigating and managing emergencies in smart cities, initial and essential part of a research project. Chapter~\ref{cap:posicionamento} presents the initial development of the risk assessment approach, such as the concept of risk zones, areas of interest and points of interest, as well as some algorithms for positioning sensing units, demonstrating the applicability of the proposed approach. Chapter~\ref{cap:cityzones} presents CityZones, a tool developed to allow any researcher to use and apply the proposed risk assessment approach. Chapter~\ref{cap:aom} presents the development of the Mitigation Area concept, expanding the area of action of the Points of Interest in the risk assessment process, as well as improving the algorithm for positioning sensing units. Chapter~\ref{cap:flood} extends the proposed approach, demonstrating the possibility of adding assessment layers, applying a layer for flood risk assessment.

\printbibliography[heading=subbibliography]

\end{refsection}
